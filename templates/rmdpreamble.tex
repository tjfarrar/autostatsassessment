
\usepackage{fancyhdr}

\pagestyle{fancy}
\fancyhf{}
%\fancypagestyle{plain}{% copies "fancy" over "plain"
	\fancyfoot[L]{Name of University (update in rmdpreamble.tex) \hskip8cm \thepage \hskip0.25em of \pageref{LastPage}}% you can add edits that won't affect "fancy" but only "plain"
%}

\usepackage{amsmath}
\usepackage{amssymb}
\usepackage{bm}
\usepackage{enumitem}
%\usepackage{Sweave}

\newlist{question}{enumerate}{9}
\setlist[question]{itemindent=0pt}
\setlist[question,1]{label=\arabic*.,ref=\arabic*}
\setlist[question,2]{label=\alph*.,ref=\alph*}
\setlist[question,3]{label=\roman*.,ref=\roman*}


\newcommand{\beq}{\begin{question}}
\newcommand{\eeq}{\end{question}}


\newenvironment{rom_enum}
{\begin{enumerate}[(i)]
  \setlength{\itemsep}{3pt}
  \setlength{\parskip}{1pt}
  \setlength{\parsep}{1pt}}
{\end{enumerate}}

\newenvironment{my_enumerate}
{\begin{enumerate}[(a)]
  \setlength{\itemsep}{1pt}
  \setlength{\parskip}{0pt}
  \setlength{\parsep}{0pt}}
{\end{enumerate}}

\lefthyphenmin=62
\righthyphenmin=62

\usepackage{booktabs}% http://ctan.org/pkg/booktabs
\newcommand{\tabitem}{~~\llap{\textbullet}~~}

\renewcommand{\rmdefault}{phv} % Arial
\renewcommand{\sfdefault}{phv} % Arial

\renewcommand{\arraystretch}{2}
\renewcommand{\headrulewidth}{0pt}

\usepackage[final]{pdfpages}
\usepackage{lastpage}

\usepackage{graphicx}
%\usepackage{tocloft} %Used to create lines and other edits to table of contents
\usepackage{mathtools}

\usepackage{babel}
\usepackage{array}
\usepackage{multirow}
%\usepackage{enumerate}
\usepackage{pifont}
\usepackage{color}

% Required by kableExtra
\usepackage{booktabs}
\usepackage{longtable}
\usepackage{array}
\usepackage{multirow}
\usepackage{wrapfig}
\usepackage{float}
\AtBeginDocument{\usepackage{colortbl}}
\usepackage{pdflscape}
\usepackage{tabu}
\usepackage{threeparttable}
\usepackage{threeparttablex}
\usepackage[normalem]{ulem}
\usepackage{makecell}


\def\Var{\mathop{\rm Var}}
\def\E{{\rm E}}
\def\Cov{\mathop{\rm Cov}}
\def\diag{\mathop{\rm diag}}
\def\tanh{\mathop{\rm tanh}}
\def\tr{\mathop{\rm tr}}
\def\sgn{\mathop{\rm sgn}}


% maxwidth is the original width if it is less than linewidth
% otherwise use linewidth (to make sure the graphics do not exceed the margin)
\makeatletter
\def\maxwidth{ %
  \ifdim\Gin@nat@width>\linewidth
    \linewidth
  \else
    \Gin@nat@width
  \fi
}
\makeatother

\definecolor{fgcolor}{rgb}{0.345, 0.345, 0.345}
\newcommand{\hlnum}[1]{\textcolor[rgb]{0.686,0.059,0.569}{#1}}%
\newcommand{\hlstr}[1]{\textcolor[rgb]{0.192,0.494,0.8}{#1}}%
\newcommand{\hlcom}[1]{\textcolor[rgb]{0.678,0.584,0.686}{\textit{#1}}}%
\newcommand{\hlopt}[1]{\textcolor[rgb]{0,0,0}{#1}}%
\newcommand{\hlstd}[1]{\textcolor[rgb]{0.345,0.345,0.345}{#1}}%
\newcommand{\hlkwa}[1]{\textcolor[rgb]{0.161,0.373,0.58}{\textbf{#1}}}%
\newcommand{\hlkwb}[1]{\textcolor[rgb]{0.69,0.353,0.396}{#1}}%
\newcommand{\hlkwc}[1]{\textcolor[rgb]{0.333,0.667,0.333}{#1}}%
\newcommand{\hlkwd}[1]{\textcolor[rgb]{0.737,0.353,0.396}{\textbf{#1}}}%
\let\hlipl\hlkwb

\usepackage{framed}
\makeatletter
\newenvironment{kframe}{%
 \def\at@end@of@kframe{}%
 \ifinner\ifhmode%
  \def\at@end@of@kframe{\end{minipage}}%
  \begin{minipage}{\columnwidth}%
 \fi\fi%
 \def\FrameCommand##1{\hskip\@totalleftmargin \hskip-\fboxsep
 \colorbox{shadecolor}{##1}\hskip-\fboxsep
     % There is no \\@totalrightmargin, so:
     \hskip-\linewidth \hskip-\@totalleftmargin \hskip\columnwidth}%
 \MakeFramed {\advance\hsize-\width
   \@totalleftmargin\z@ \linewidth\hsize
   \@setminipage}}%
 {\par\unskip\endMakeFramed%
 \at@end@of@kframe}
\makeatother

\definecolor{shadecolor}{rgb}{.97, .97, .97}
\definecolor{messagecolor}{rgb}{0, 0, 0}
\definecolor{warningcolor}{rgb}{1, 0, 1}
\definecolor{errorcolor}{rgb}{1, 0, 0}
\newenvironment{knitrout}{}{} % an empty environment to be redefined in TeX

\usepackage{alltt}
\IfFileExists{upquote.sty}{\usepackage{upquote}}{}


\usepackage[colorlinks=true,linkcolor=blue]{hyperref}
\floatplacement{figure}{H}
\floatplacement{table}{H}


\newenvironment{changemargin}[2]{%
  \begin{list}{}{%
    \setlength{\topsep}{0pt}%
    \setlength{\leftmargin}{#1}%
      \setlength{\rightmargin}{#2}%
        \setlength{\listparindent}{\parindent}%
        \setlength{\itemindent}{\parindent}%
        \setlength{\parsep}{\parskip}%
      }%
      \item[]}{\end{list}}


    \newcommand{\tick}{\ding{110} }
    \newcommand{\halftick}{\ding{115} }
    
    \hypersetup{pdfstartview={XYZ null null 1.00}}
    
% sascaption is needed for saspdf engine for two-way frequency tables in R Markdown

\iffalse
\usepackage{sasmarkdown}
\def\sasCaption{%
    \normalsize%
    \bfseries%
    \sffamily%
    \color[rgb]{0,0,0}%
    \columncolor[rgb]{0.87,0.87,0.87}%
  }
\fi
